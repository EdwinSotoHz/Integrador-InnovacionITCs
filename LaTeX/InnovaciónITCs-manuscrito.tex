\documentclass{IEEEcsmag}

\usepackage[colorlinks,urlcolor=blue,linkcolor=blue,citecolor=blue]{hyperref}
\expandafter\def\expandafter\UrlBreaks\expandafter{\UrlBreaks\do\/\do\*\do\-\do\~\do\'\do\"\do\-}
\usepackage{upmath,color}

\usepackage[spanish]{babel}
%\usepackage[latin1]{inputenc}
\usepackage[utf8]{inputenc}  

\jvol{1}
\jnum{1}
\paper{1}
\jmonth{Noviembre}
\jname{ITICs letters}
\jtitle{Proyectos Integradores}
\pubyear{2023}
\usepackage{cite}
\usepackage{amsmath,amssymb,amsfonts}
\usepackage{algorithmic}
\usepackage{graphicx}
\usepackage{textcomp}
\usepackage{xcolor}
\usepackage{listings}

\newtheorem{theorem}{Theorem}
\newtheorem{lemma}{Lemma}


\setcounter{secnumdepth}{0}

\begin{document}

%imprimir código
\lstnewenvironment{javaCode}[1][]
{\lstset{
    language=Java,
    basicstyle=\scriptsize\ttfamily,
    numbers=none, % Modificado: quitar los números de línea
    keywordstyle=\color{blue},
    commentstyle=\color{gray},
    stringstyle=\color{purple},
    breaklines=true,
    breakatwhitespace=true,
    tabsize=4,
    showspaces=false,
    showstringspaces=false,
    frame=single,
    captionpos=b,
    floatplacement=!h,
    #1
}}
{}


\sptitle{Proyecto Integrador de Primer Semestre}

\title{Software de resolución de problemas de Ingeniería }


\author{Ángeles Martínez Dilan Emir}
\affil{Instituto Tecnológico Superior del Occidente del Estado de Hidalgo, Mixquiahuala, Hgo., 42700, Mexico}

\author{Soto Hernandez Edwin salvador}
\affil{Instituto Tecnológico Superior del Occidente del Estado de Hidalgo, Mixquiahuala, Hgo., 42700, Mexico}

\author{Pérez Ortiz Karen }
\affil{Instituto Tecnológico Superior del Occidente del Estado de Hidalgo, Mixquiahuala, Hgo., 42700, Mexico}

\author{Ortiz Escamilla José María}
\affil{Instituto Tecnológico Superior del Occidente del Estado de Hidalgo, Mixquiahuala, Hgo., 42700, Mexico}

\author{Bernal Franco Lizbeth de jesús}
\affil{Instituto Tecnológico Superior del Occidente del Estado de Hidalgo, Mixquiahuala, Hgo., 42700, Mexico}

%\author{Third Author III}
%\affil{Institute, City, (State), Postal Code, Country}

\markboth{ITSOEH/ITICS/PROYECTO INTEGRADOR PRIMER SEMESTRE}{THEME/FEATURE/DEPARTMENT}

\begin{abstract}
Un resumen (abstract) es un párrafo único que resume los aspectos importantes del manuscrito. A menudo indica si el manuscrito es un informe de un trabajo nuevo, una revisión o una descripción general, o una combinación de ambos. No cite referencias en el resumen. Este tipo de documento debe incluir contenido propiedad de los autores; es decir, no debe contener contenido de otras fuentes, ademas la redacción debe  estar dirigida a un tipo de lector técnico general. Este archivo se encuentra disponible en \href{https://github.com/fcuadrosgithub/integrador-primero.git}{https://github.com/fcuadrosgithub/integrador-primero.git}.
\end{abstract}

\maketitle
\chapteri{E}ste proyecto integral se centra en el desarrollo de un software diseñado para abordar y resolver problemas de ingeniería mediante la programación en Java. La metodología empleada sigue un enfoque estructurado que consta de seis etapas fundamentales: descripción del problema, definición de soluciones, diseño, desarrollo, depuración y pruebas, así como documentación.

El propósito esencial de este proyecto es ofrecer soluciones eficaces y claras para diversos desafíos matemáticos y lógicos, en esta investigación, se abordarán problemas específicos que involucran no solo cálculos geométricos y álgebra, sino también conceptos provenientes del cálculo y matemáticas discretas, enriqueciendo así la aplicación práctica del software desarrollado. A lo largo de este manuscrito, se detallará la resolución de seis problemas específicos, y los resultados obtenidos se reflejarán en programas en Java que abordan de manera satisfactoria cada uno de los desafíos planteados.

En las secciones siguientes, se presentará un análisis detallado de la resolución de cada problema, destacando las contribuciones clave del software desarrollado. Las conclusiones resumirán los hallazgos más significativos, discutiendo posibles extensiones o mejoras en futuras investigaciones. Además, se expresarán agradecimientos, se incluirán referencias y se proporcionarán las biografías de los integrantes del proyecto para contextualizar la investigación. 

\clearpage

\section{Resolución Problema 1} 
\subsection{Problema:}
Dados 2 puntos $A \mbox{ y } B$ con coordenadas $x_{1}, y_{1}$ y $x_{2}, y_{2}$  respectivamente. Regresar la ecuación de la recta y el ángulo interno $\alpha$ que se forma entre el eje horizontal y la recta. 

\subsection{\textbf{Descripción del problema:}}

\subsection{\textbf{Definición de solución:}}

\subsection{\textbf{Diseño de la solución:}}

\subsection{\textbf{Desarrollo de la solución:}}

\subsection{\textbf{Depuración y pruebas:}}
\clearpage

\section{Resolución Problema 2} 
\subsection{Problema:}
Dada una ecuación cuadratica regresar los valores de las raíces en caso de que estén sobre el conjunto de los números reales, en caso contrario indicar que la solución esta en el conjunto de los números complejos. 

\subsection{\textbf{Descripción del problema:}}
Se debe determinar si las raíces de una ecuación cuadrática 
son reales o complejas. Si son reales, se calculan y 
se proporcionan los valores. Si son complejas, se 
indica que la solución está en el conjunto de los números complejos.

\subsection{\textbf{Definición de solución:}}
Este proyecto aborda el problema de resolver ecuaciones cuadráticas, identificando si sus raíces son reales o complejas. 
Para determinar la naturaleza de las raíces de una ecuación cuadrática de la forma  (\(ax^2 + bx + c = 0\)), se evalúa el discriminante. Si es negativo, las raíces son complejas. De lo contrario, se calculan utilizando la fórmula cuadrática estándar:

\[ x = \frac{-b \pm \sqrt{b^2 - 4ac}}{2a} \]

Donde:
\begin{itemize}
    \item \(a\), \(b\), y \(c\) son los coeficientes de la ecuación cuadrática.
    \item El valor del discriminante determina si las raíces son reales o complejas y proporciona información sobre la cantidad de raíces distintas.
\end{itemize}


\begin{figure}[h!]
    \centering
    \includegraphics[width=0.6\linewidth]{latex-imagenes/graficaP1.png}
    \caption{Representación de la conversión}
    \label{fig: Grafica Ecuacion Recta}
\end{figure}

\subsection{\textbf{Diseño de la solución:}}

\subsection{\textbf{Desarrollo de la solución:}}

\subsection{\textbf{Depuración y pruebas:}}
\clearpage

\section{Resolución Problema 3}
\subsection{Problema:}
Dada una circunferencia con centro en el punto $C$ con coordenadas $(x_{1}, y_{1})$ y radio $r$, evaluar si un punto $T$ con coordenadas $(x_{2}, y_{2})$ esta dentro del area de la circunferencia.

\subsection{\textbf{Descripción del problema:}}

\subsection{\textbf{Definición de solución:}}

\subsection{\textbf{Diseño de la solución:}}

\subsection{\textbf{Desarrollo de la solución:}}

\subsection{\textbf{Depuración y pruebas:}}
\clearpage

\section{Resolución Problema 4}
\subsection{Problema:}
Dado un numero decimal entero positivo o negativo regresar su equivalente en binario.


\subsection{\textbf{Descripción del problema:}}

\subsection{\textbf{Definición de solución:}}

\subsection{\textbf{Diseño de la solución:}}

\subsection{\textbf{Desarrollo de la solución:}}

\subsection{\textbf{Depuración y pruebas:}}
\clearpage

\section{Resolución Problema 5}
\subsection{Problema:}
Dado un numero binario de $n$ bits regresar su equivalente en decimal.

\subsection{\textbf{Descripción del problema:}}

\subsection{\textbf{Definición de solución:}}

\subsection{\textbf{Diseño de la solución:}}

\subsection{\textbf{Desarrollo de la solución:}}

\subsection{\textbf{Depuración y pruebas:}}
\clearpage

\section{Resolución Problema 6}
\subsection{Problema:}
\item Dada una tabla de verdad de $n$ bits generar la expresión booleana que genere de manera fidedigna las salidas de esta tabla.

\subsection{\textbf{Descripción del problema:}}

\subsection{\textbf{Definición de solución:}}

\subsection{\textbf{Diseño de la solución:}}

\subsection{\textbf{Desarrollo de la solución:}}

\subsection{\textbf{Depuración y pruebas:}}
\clearpage


\section{CONCLUSION}
El manuscrito debe incluir direcciones futuras de la investigación. Se recomienda encarecidamente a los autores que no hagan referencia a varias figuras o tablas en la conclusión; estos deben mencionarse en el cuerpo del artículo.
\vspace*{-8pt}


\section{AGRADECIMIENTOS}
Esta sección es opcional. Si los autores creen necesario agradecer a alguien por haber aportado al desarrollo de su proyecto integrador de alguna u otra forma, esta sección esta destinada para esto.


\def\refname{REFERENCES}

\begin{thebibliography}{1}

\bibitem{AA1}
G. M. Amdahl, G. A. Blaauw, and F. P. Brooks, ``Architecture of the IBM System/360,'' {\it IBM J. Res. Dev}., vol. 8, no. 2, pp. 87--101, 1964. (Journal)

\bibitem{BB1}  
Cunoticias. (s. f.). Complemento a 1 (uno) con ejemplos. https://www.cunoticias.com/internet/complemento-a-1-uno-con-ejemplos.php

\bibitem{CC1}
Johnsonbaugh, R. (s. f.). MATEMÁTICAS DISCRETAS. Pearson Educación.

\bibitem{DD1}
Fernández, M. Y. A., Hernández, Y. J. S., & Montaña, M. M. (s. f.). Matemáticas Discretas:: Con un enfoque desde la ingeniería y ciencias sociales - Conceptos básicos. Editorial de la Universidad Pedagógica y Tecnológica de Colombia - UPTC.

\bibitem{BB1}
Sección 5: Soto, J. A. - GEEKNETIC. (Cómo convertir binario en decimal paso a paso). [Online]. Available: {https://www.geeknetic.es/Guia/2667/Como-convertir-binario-en-decimal-paso-a-paso.html} (URL)

\bibitem{CC1}
Sección 5: Escobar, B., & Perfil, V. T. mi. - Blogspot.com.. (Algoritmo para pasar de Binario a Decimal.). [Online]. Available: {http://mensajeseintereses.blogspot.com/2011/10/2-algoritmo-para-pasar-de-binario.html} (URL)

\end{thebibliography}\vspace*{-8pt}


\begin{IEEEbiography}{Cuadros Romero Francisco Javier}{\,}Todas las biografías se limitan a un párrafo y deben ser muy sintéticas. Se puede agregar la carrera en la cual el estudiante esta enrolado. Se pueden mencionar los 3 intereses principales del estudiante. Asi como su aspiración en el corto y mediano plazo. Al final de la biografia de cada estudiante se debe agregar el enlace a su pagina personal en Github: 
%\vadjust{\vfill\pagebreak}
\end{IEEEbiography}

\begin{IEEEbiography}{Karen Perez Ortiz}{\,}Originaria de Mixquiahuala de Juarez, es una estudiante de Ingeniería en Tecnologías de la Información y Comunicaciones cuyo interés por la tecnología floreció durante la secundaria, gracias a una reveladora convención tecnológica. Fascinada por la versatilidad creativa que una computadora puede ofrecer, se sumerge en la exploración de diversas ramas tecnológicas. Más allá de su dedicación académica, disfruta de series animadas, la astronomía y seguir a streamers. Con metas ambiciosas, no solo aspira a concluir la universidad, sino también a dejar un legado significativo en el campo. Su enfoque se centra en el desarrollo web con énfasis en Experiencia de Usuario (UX), y su entusiasmo por la Inteligencia Artificial y el Aprendizaje Automático la impulsa a profundizar constantemente en estos campos, motivada por la perspectiva de contribuir al avance tecnológico. https://karenperezor.github.io/.
\end{IEEEbiography}
\begin{IEEEbiography}{Neri Pérez Giovany Humberto}{\,}es un estudiante de la ingeniería en Tecnologías de la Información y Comunicaciones con una pasión desbordante por los videojuegos, el anime y los ``corridos tumbados''. Nacido y criado en  Tetepango, desde temprana edad mostró un gran interés por la tecnología y los avances en el campo de la informática. Aunque sus intereses pueden parecer diversos, Giovany encuentra inspiración en la creatividad y la narrativa tanto de los videojuegos como del anime. Estos medios le han enseñado la importancia de la perseverancia, la resolución de problemas y el trabajo en equipo. El objetivo principal de Giovany es completar sus estudios universitarios en ITICs, con especialización en ciberseguridad. Sueña con trabajar en el campo de la ciberseguridad, protegiendo sistemas e información vital de ataques cibernéticos y contribuyendo así a la seguridad digital de las organizaciones.
\end{IEEEbiography}

\end{document}


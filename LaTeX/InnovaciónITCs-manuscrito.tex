\documentclass{IEEEcsmag}

\usepackage[colorlinks,urlcolor=blue,linkcolor=blue,citecolor=blue]{hyperref}
\expandafter\def\expandafter\UrlBreaks\expandafter{\UrlBreaks\do\/\do\*\do\-\do\~\do\'\do\"\do\-}
\usepackage{upmath,color}

\usepackage[spanish]{babel}
%\usepackage[latin1]{inputenc}
\usepackage[utf8]{inputenc}  

\jvol{1}
\jnum{1}
\paper{1}
\jmonth{Noviembre}
\jname{ITICs letters}
\jtitle{Proyectos Integradores}
\pubyear{2023}
\usepackage{cite}
\usepackage{amsmath,amssymb,amsfonts}
\usepackage{algorithmic}
\usepackage{graphicx}
\usepackage{textcomp}
\usepackage{xcolor}
\usepackage{listings}

\newtheorem{theorem}{Theorem}
\newtheorem{lemma}{Lemma}


\setcounter{secnumdepth}{0}

\begin{document}

%imprimir código
\lstnewenvironment{javaCode}[1][]
{\lstset{
    language=Java,
    basicstyle=\scriptsize\ttfamily,
    numbers=none, % Modificado: quitar los números de línea
    keywordstyle=\color{blue},
    commentstyle=\color{gray},
    stringstyle=\color{purple},
    breaklines=true,
    breakatwhitespace=true,
    tabsize=4,
    showspaces=false,
    showstringspaces=false,
    frame=single,
    captionpos=b,
    floatplacement=!h,
    #1
}}
{}


\sptitle{Proyecto Integrador de Primer Semestre}

\title{Software de resolución de problemas de Ingeniería }


\author{Soto Hernandez Edwin salvador}
\affil{Instituto Tecnológico Superior del Occidente del Estado de Hidalgo, Tlahuelilpan, Hgo., 42790, Mexico}

\author{Pérez Ortiz Karen }
\affil{Instituto Tecnológico Superior del Occidente del Estado de Hidalgo, Mixquiahuala, Hgo., 42700, Mexico}

\author{Ortiz Escamilla José María}
\affil{Instituto Tecnológico Superior del Occidente del Estado de Hidalgo, Mixquiahuala, Hgo., 42700, Mexico}

\author{Bernal Franco Lizbeth de jesús}
\affil{Instituto Tecnológico Superior del Occidente del Estado de Hidalgo, Mixquiahuala, Hgo., 42700, Mexico}

\author{Ángeles Martínez Dilan Emir}
\affil{Instituto Tecnológico Superior del Occidente del Estado de Hidalgo, Mixquiahuala, Hgo., 42700, Mexico}

%\author{Third Author III}
%\affil{Institute, City, (State), Postal Code, Country}

\markboth{ITSOEH/ITICS/PROYECTO INTEGRADOR PRIMER SEMESTRE}{THEME/FEATURE/DEPARTMENT}

\begin{abstract}
Este manuscrito presenta el desarrollo de un software en Java destinado a resolver problemas de ingeniería. Adoptando una metodología estructurada de seis etapas, desde la descripción del problema hasta la documentación, el proyecto aborda desafíos específicos en geometría, álgebra, trigonometría y lógica booleana. Los resultados, expresados en programas Java, ofrecen soluciones efectivas para seis problemas distintos. 
\end{abstract}

\maketitle
\chapteri{E}ste proyecto se centra abordar y resolver problemas de ingeniería mediante la programación en Java. La metodología empleada sigue un enfoque estructurado que consta de seis etapas fundamentales: descripción del problema, definición de soluciones, diseño, desarrollo, depuración y pruebas, así como documentación.

El propósito esencial de este proyecto es ofrecer soluciones eficaces y claras para diversos desafíos matemáticos y lógicos, en esta investigación, se abordarán problemas específicos que involucran no solo cálculos geométricos y álgebra, sino también conceptos provenientes del cálculo y matemáticas discretas, enriqueciendo así la aplicación práctica del software desarrollado. 
En las secciones siguientes, se presentará un análisis detallado de la resolución de cada problema, destacando las contribuciones clave del software desarrollado. La conclusión resumirán los hallazgos más significativos, discutiendo posibles extensiones o mejoras en futuras investigaciones. Además, se expresarán agradecimientos, se incluirán referencias y se proporcionarán las biografías de los integrantes del proyecto para contextualizar la investigación. 

\clearpage

\section{Resolución Problema 1} 
\subsection{Problema:}
Dados 2 puntos $A \mbox{ y } B$ con coordenadas $x_{1}, y_{1}$ y $x_{2}, y_{2}$  respectivamente. Regresar la ecuación de la recta y el ángulo interno $\alpha$ que se forma entre el eje horizontal y la recta. 

\subsection{\textbf{Descripción del problema:}}

\subsection{\textbf{Definición de solución:}}

\subsection{\textbf{Diseño de la solución:}}

\subsection{\textbf{Desarrollo de la solución:}}

\subsection{\textbf{Depuración y pruebas:}}
\clearpage

\section{Resolución Problema 2} 
\subsection{Problema:}
Dada una ecuación cuadratica regresar los valores de las raíces en caso de que estén sobre el conjunto de los números reales, en caso contrario indicar que la solución esta en el conjunto de los números complejos. 

\subsection{\textbf{Descripción del problema:}}
Se debe determinar si las raíces de una ecuación cuadrática 
son reales o complejas. Si son reales, se calculan y 
se proporcionan los valores. Si son complejas, se 
indica que la solución está en el conjunto de los números complejos.

\subsection{\textbf{Definición de solución:}}
Este proyecto aborda el problema de resolver ecuaciones cuadráticas, identificando si sus raíces son reales o complejas. 
Para determinar la naturaleza de las raíces de una ecuación cuadrática de la forma  (\(ax^2 + bx + c = 0\)), se evalúa el discriminante. Si es negativo, las raíces son complejas. De lo contrario, se calculan utilizando la fórmula cuadrática estándar:

\[ x = \frac{-b \pm \sqrt{b^2 - 4ac}}{2a} \]

Donde:
\begin{itemize}
    \item \(a\), \(b\), y \(c\) son los coeficientes de la ecuación cuadrática.
    \item El valor del discriminante determina si las raíces son reales o complejas y proporciona información sobre la cantidad de raíces distintas.
\end{itemize}


\begin{figure}[h!]
    \centering
    \includegraphics[width=0.6\linewidth]{latex-imagenes/graficaP1.png}
    \caption{Representación de la conversión}
    \label{fig: Grafica Ecuacion Recta}
\end{figure}

\subsection{\textbf{Diseño de la solución:}}

\subsection{\textbf{Desarrollo de la solución:}}

\subsection{\textbf{Depuración y pruebas:}}
\clearpage

\section{Resolución Problema 3}
\subsection{Problema:}
Dada una circunferencia con centro en el punto $C$ con coordenadas $(x_{1}, y_{1})$ y radio $r$, evaluar si un punto $T$ con coordenadas $(x_{2}, y_{2})$ esta dentro del area de la circunferencia.

\subsection{\textbf{Descripción del problema:}}

\subsection{\textbf{Definición de solución:}}

\subsection{\textbf{Diseño de la solución:}}

\subsection{\textbf{Desarrollo de la solución:}}

\subsection{\textbf{Depuración y pruebas:}}
\clearpage

\section{Resolución Problema 4}
\subsection{Problema:}
Dado un numero decimal entero positivo o negativo regresar su equivalente en binario.


\subsection{\textbf{Descripción del problema:}}

\subsection{\textbf{Definición de solución:}}

\subsection{\textbf{Diseño de la solución:}}

\subsection{\textbf{Desarrollo de la solución:}}

\subsection{\textbf{Depuración y pruebas:}}
\clearpage

\section{Resolución Problema 5}
\subsection{Problema:}
Dado un numero binario de $n$ bits regresar su equivalente en decimal.

\subsection{\textbf{Descripción del problema:}}

\subsection{\textbf{Definición de solución:}}

\subsection{\textbf{Diseño de la solución:}}

\subsection{\textbf{Desarrollo de la solución:}}

\subsection{\textbf{Depuración y pruebas:}}
\clearpage

\section{Resolución Problema 6}
\subsection{Problema:}
\item Dada una tabla de verdad de $n$ bits generar la expresión booleana que genere de manera fidedigna las salidas de esta tabla.

\subsection{\textbf{Descripción del problema:}}

\subsection{\textbf{Definición de solución:}}

\subsection{\textbf{Diseño de la solución:}}

\subsection{\textbf{Desarrollo de la solución:}}

\subsection{\textbf{Depuración y pruebas:}}
\clearpage


\section{CONCLUSION}
En este proyecto, hemos alcanzado con éxito la implementación de un software en Java para abordar problemas de ingeniería, destacando la eficacia de la metodología estructurada de seis etapas (6D). La eficiencia de este enfoque se evidencia en los resultados obtenidos, los cuales no solo proporcionan soluciones prácticas y claras, sino que también resaltan la versatilidad del software al integrar conceptos de álgebra, cálculo y lógica booleana. 
Más allá de la creación del software, la verdadera trascendencia de este proyecto radica en el proceso mismo de construir soluciones mediante la programación. La sinergia entre conceptos de geometría, álgebra, cálculo y matemáticas discretas no solo ha sido evidente en las soluciones propuestas, sino que también destaca la importancia continua en la resolución de problemas complejos en la ingeniería moderna. Este proyecto no solo abre nuevas posibilidades para futuras investigaciones en la intersección de programación y disciplinas matemáticas, sino que también refuerza la idea de que el método de resolución es tan valioso como el resultado mismo.
\vspace*{-8pt}


\section{AGRADECIMIENTOS}
Queremos expresar nuestro sincero agradecimiento a aquellos que contribuyeron de manera significativa al desarrollo de nuestro proyecto integrador. En particular, deseamos reconocer y agradecer a los siguientes docentes por su invaluable apoyo y orientación:
Agradecemos al profesor Neri Pérez Giovany Humberto por su dedicación y asesoramiento en el aprendizaje de LaTeX, así como por su labor en la corrección de la codificación y la explicación detallada de los problemas de cálculo que surgieron durante el desarrollo de nuestro proyecto.
Extendemos nuestro agradecimiento al profesor Agustín Soto Arista, cuya colaboración fue fundamental en las pruebas de los problemas relacionados con Matemáticas discretas. Su expertise y disposición para ayudarnos fueron esenciales para mejorar la calidad de nuestro trabajo.
Además, agradecemos a la profesora Eunice Santiago Manzano por su valiosa asistencia en los ensayos de exposición y su dedicación en la depuración de problemas. Su orientación y perspectiva fueron esenciales para mejorar nuestra presentación y abordar eficientemente los problemas que encontramos.
Su guía y enseñanzas han dejado una huella significativa en nuestro aprendizaje y desarrollo académico, muchas gracias por su tiempo, paciencia y dedicación.

\def\refname{REFERENCES}

\begin{thebibliography}{1}
\bibitem{BB1}  
Cunoticias. (s. f.). Complemento a 1 (uno) con ejemplos. https://www.cunoticias.com/internet/complemento-a-1-uno-con-ejemplos.php

\bibitem{BB1}  
Clases particulares y Profesores particulares.(2023, May 10).(Fórmulas para calcular la distancia entre dos puntos). [Online]. Available: {https://www.tusclasesparticulares.com/blog/como-calcular-distancia-entre-dos-puntos} (URL)

\bibitem{CC1}
Johnsonbaugh, R. (s. f.). MATEMÁTICAS DISCRETAS. Pearson Educación.

\bibitem{DD1}
Fernández, M. Y. A., Hernández, Y. J. S., & Montaña, M. M. (s. f.). Matemáticas Discretas:: Con un enfoque desde la ingeniería y ciencias sociales - Conceptos básicos. Editorial de la Universidad Pedagógica y Tecnológica de Colombia - UPTC.

\bibitem{BB1}
Sección 5: Soto, J. A. - GEEKNETIC. (Cómo convertir binario en decimal paso a paso). [Online]. Available: {https://www.geeknetic.es/Guia/2667/Como-convertir-binario-en-decimal-paso-a-paso.html} (URL)

\bibitem{CC1}
Sección 5: Escobar, B., & Perfil, V. T. mi. - Blogspot.com.. (Algoritmo para pasar de Binario a Decimal.). [Online]. Available: {http://mensajeseintereses.blogspot.com/2011/10/2-algoritmo-para-pasar-de-binario.html} (URL)


\bibitem{BB1}  
All About Circuits. (2023). Conversión de tablas de verdad en expresiones booleanas. En Libro de texto de electrónica. https://www.allaboutcircuits.com/textbook/digital/chpt-7/converting-truth-tables-boolean-expressions/

\bibitem{BB1}  
Xiang, Y., Dalchau, N., & Wang, B. (2018). Scaling up genetic circuit design for cellular computing: advances and prospects. Natural computing, 17(4), 833-853.

\bibitem{CC1}  
Cunoticias. (s. f.). Complemento a 1 (uno) con ejemplos. https://www.cunoticias.com/internet/complemento-a-1-uno-con-ejemplos.php

\bibitem{BB1}
/u/school_isa. (2017, septiembre 29). \\
FORMÚLA GENERAL PARA ECUACIONES CUADRÁTICAS.\\
GeoGebra. https://www.geogebra.org/m/GYXrzYEF

\end{thebibliography}\vspace*{-8pt}



\begin{IEEEbiography}{Karen Perez Ortiz}{\,}Es una estudiante de la Ingeniería en Tecnologías de la Información y Comunicaciones, sus intereses son las series animadas, la astronomía y ver streamers. Nació en Mixquiahuala de Juarez, aunque desconocía las ramas de la tecnología no fue desde la secundaria que a partir de una convención de tecnología fue que se intereso en estudiar en las tecnologias. La forma en que a partir de una computadora se pueden crear diversas cosas se volvio fasivanate para ella, Sus metas no solo es concluir la universidad si no que además quiere dejar algo significativo que beneficie a las personas que siempre la apoyaron, sus principales intereses son desarrollo web en Experiencia de Usuario, así Inteligencia Artificial y Aprendizaje Automático la forma en que le explicaron sobre este tema la animo a estudiar mas y mas de ello. (link de mi github: https://karenperezor.github.io/.)
\end{IEEEbiography}
\begin{IEEEbiography}{Edwin Salvador Soto Hernández}{\,}Edwin Salvador Soto Hernández, de 18 años y originario del estado de Querétaro, se encuentra inmerso en la ingeniería en Tecnologías de la Información y Comunicaciones en el ITSOEH, ubicado en Tlahuelilpan, Hidalgo. Desde una edad temprana, demostró un ferviente interés por la tecnología e informática, y ahora, tras dos años en este estado, ha consolidado sus estudios de nivel medio superior. Su exploración en diversas ramas tecnológicas abarca el desarrollo web, desarrollo móvil y UX/UI, mientras que sus principales intereses personales incluyen el modelado 3D, disfrutar de series animadas y videojuegos. A un corto plazo, aspira a concluir la universidad y especializarse en áreas que le apasionan, como ciberseguridad, redes y desarrollo de software. A largo plazo, su ambición es colaborar en proyectos grandes que tengan relevancia en el futuro, contribuyendo así al avance tecnológico y tener relevancia en el campo. Visita mi \href{https://edwinsotohz.github.io/}{página personal en GitHub}.
\end{IEEEbiography}
\begin{IEEEbiography}{José María Ortiz Escamilla}{\,}Originario de Mixquiahuala de Juárez, Hgo. Es un estudiante de Ingeniería en Tecnologías de la Información y Comunicaciones. El cual desde una edad temprana mostró curiosidad por las tecnologías y generando un grande interés por ellas desde su educación de nivel medio superior al estudiar la especialidad de programación, muestra una fascinación por la música y el anime de todo tipo, le apasiona ir al gimnasio, ya que mejora el cómo se siente con el mismo y lo hace sentir capaz de lograr lo que se proponga, lo que desea conseguir a largo plazo desea poder trabajar en empresas que tengan una gran relevancia en el área de la ciberseguridad o el desarrollo de software teniendo la oportunidad de dejar huella en el área de la informática.Visita mi \href{https://joseoe.github.io/}{página personal en GitHub}.
\end{IEEEbiography}
\begin{IEEEbiography}{Bernal Franco Lizbeth de Jesús}{\,}es una estudainte que cursa el primer semestre de la ingeniería en Tecnologías de la Información y Comunicaciones con muchas pasiones y metas persornales que la impulsan a seguir. Originaria de Mixquiahula de Juarez, desde el segundo año de bachirerato ella desarrollo un interés hacia las redes, desarrollo web y ciberseguridad, esto fue un impulso para que ella viera la posibilidad de estudiar o convertirse en una especialista en la tecnologia, su gran meta es terminar su carrera y convertirse en toda una profesional y trabajar en proyectos a un mejor crear un proyecto. Visita mi \href{https://lizbernal.github.io/}{página personal en GitHub}.
\end{IEEEbiography}
\end{document}
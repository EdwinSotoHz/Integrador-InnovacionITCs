\documentclass{IEEEcsmag}

\usepackage[colorlinks,urlcolor=blue,linkcolor=blue,citecolor=blue]{hyperref}
\expandafter\def\expandafter\UrlBreaks\expandafter{\UrlBreaks\do\/\do\*\do\-\do\~\do\'\do\"\do\-}
\usepackage{upmath,color}

\usepackage[spanish]{babel}
%\usepackage[latin1]{inputenc}
\usepackage[utf8]{inputenc}  

\jvol{1}
\jnum{1}
\paper{1}
\jmonth{Noviembre}
\jname{ITICs letters}
\jtitle{Proyectos Integradores}
\pubyear{2023}
\usepackage{cite}
\usepackage{amsmath,amssymb,amsfonts}
\usepackage{algorithmic}
\usepackage{graphicx}
\usepackage{textcomp}
\usepackage{xcolor}
\usepackage{listings}

\newtheorem{theorem}{Theorem}
\newtheorem{lemma}{Lemma}


\setcounter{secnumdepth}{0}

\begin{document}

%imprimir código
\lstnewenvironment{javaCode}[1][]
{\lstset{
    language=Java,
    basicstyle=\scriptsize\ttfamily,
    numbers=none, % Modificado: quitar los números de línea
    keywordstyle=\color{blue},
    commentstyle=\color{gray},
    stringstyle=\color{purple},
    breaklines=true,
    breakatwhitespace=true,
    tabsize=4,
    showspaces=false,
    showstringspaces=false,
    frame=single,
    captionpos=b,
    floatplacement=!h,
    #1
}}
{}


\sptitle{Proyecto Integrador de Primer Semestre}

\title{Software de resolución de problemas de Ingeniería }

\author{Cuadros Romero Francisco Javier}
\affil{Instituto Tecnológico Superior del Occidente del Estado de Hidalgo, Mixquiahuala, Hgo., 42700, Mexico}

\author{Neri Pérez Giovany Humberto}
\affil{Instituto Tecnológico Superior del Occidente del Estado de Hidalgo, Mixquiahuala, Hgo., 42700, Mexico}

%\author{Third Author III}
%\affil{Institute, City, (State), Postal Code, Country}

\markboth{ITSOEH/ITICS/PROYECTO INTEGRADOR PRIMER SEMESTRE}{THEME/FEATURE/DEPARTMENT}

\begin{abstract}
Un resumen (abstract) es un párrafo único que resume los aspectos importantes del manuscrito. A menudo indica si el manuscrito es un informe de un trabajo nuevo, una revisión o una descripción general, o una combinación de ambos. No cite referencias en el resumen. Este tipo de documento debe incluir contenido propiedad de los autores; es decir, no debe contener contenido de otras fuentes, ademas la redacción debe  estar dirigida a un tipo de lector técnico general. Este archivo se encuentra disponible en \href{https://github.com/fcuadrosgithub/integrador-primero.git}{https://github.com/fcuadrosgithub/integrador-primero.git}.
\end{abstract}

\maketitle
\chapteri{L}a introducción debe proporcionar información general (incluidas referencias relevantes) y debe indicar el propósito del manuscrito. En esta sección describa de manera clara y precisa el objetivo del proyecto integrador, la metodología que piensa usar y los resultados obtenidos de manera muy general. Dentro de esta sección puede citar trabajos relevantes de otros si lo cree necesario.

Esta sección debe dar un panorama muy general al lector de cual es el problema a resolver, que metodología utilizó para dar solución al problema y cuales fueron los resultados obtenidos. 

La redacción del manuscrito debe ser en tercera persona y queda estrictamente prohibido el uso de palabras coloquiales o Español informal. En lugar de esto utilice un lenguaje formal que el mayor numero de personas pueda entender.
\clearpage

\section{Resolución Problema 1} 
\subsection{Problema:}
Dados 2 puntos $A \mbox{ y } B$ con coordenadas $x_{1}, y_{1}$ y $x_{2}, y_{2}$  respectivamente. Regresar la ecuación de la recta y el ángulo interno $\alpha$ que se forma entre el eje horizontal y la recta. 

\subsection{\textbf{Descripción del problema:}}

\subsection{\textbf{Definición de solución:}}

\subsection{\textbf{Diseño de la solución:}}

\subsection{\textbf{Desarrollo de la solución:}}

\subsection{\textbf{Depuración y pruebas:}}
\clearpage

\section{Resolución Problema 2} 
\subsection{Problema:}
Dada una ecuación cuadratica regresar los valores de las raíces en caso de que estén sobre el conjunto de los números reales, en caso contrario indicar que la solución esta en el conjunto de los números complejos. 

\subsection{\textbf{Descripción del problema:}}
Se debe determinar si las raíces de una ecuación cuadrática 
son reales o complejas. Si son reales, se calculan y 
se proporcionan los valores. Si son complejas, se 
indica que la solución está en el conjunto de los números complejos.

\subsection{\textbf{Definición de solución:}}
Este proyecto aborda el problema de resolver ecuaciones cuadráticas, identificando si sus raíces son reales o complejas. 
Para determinar la naturaleza de las raíces de una ecuación cuadrática de la forma  (\(ax^2 + bx + c = 0\)), se evalúa el discriminante. Si es negativo, las raíces son complejas. De lo contrario, se calculan utilizando la fórmula cuadrática estándar:

\[ x = \frac{-b \pm \sqrt{b^2 - 4ac}}{2a} \]

Donde:
\begin{itemize}
    \item \(a\), \(b\), y \(c\) son los coeficientes de la ecuación cuadrática.
    \item El valor del discriminante determina si las raíces son reales o complejas y proporciona información sobre la cantidad de raíces distintas.
\end{itemize}


\begin{figure}[h!]
    \centering
    \includegraphics[width=0.6\linewidth]{latex-imagenes/graficaP1.png}
    \caption{Representación de la conversión}
    \label{fig: Grafica Ecuacion Recta}
\end{figure}

\subsection{\textbf{Diseño de la solución:}}

\subsection{\textbf{Desarrollo de la solución:}}

\subsection{\textbf{Depuración y pruebas:}}
\clearpage

\section{Resolución Problema 3}
\subsection{Problema:}
Dada una circunferencia con centro en el punto $C$ con coordenadas $(x_{1}, y_{1})$ y radio $r$, evaluar si un punto $T$ con coordenadas $(x_{2}, y_{2})$ esta dentro del area de la circunferencia.

\subsection{\textbf{Descripción del problema:}}

\subsection{\textbf{Definición de solución:}}

\subsection{\textbf{Diseño de la solución:}}

\subsection{\textbf{Desarrollo de la solución:}}

\subsection{\textbf{Depuración y pruebas:}}
\clearpage

\section{Resolución Problema 4}
\subsection{Problema:}
Dado un numero decimal entero positivo o negativo regresar su equivalente en binario.


\subsection{\textbf{Descripción del problema:}}

\subsection{\textbf{Definición de solución:}}

\subsection{\textbf{Diseño de la solución:}}

\subsection{\textbf{Desarrollo de la solución:}}

\subsection{\textbf{Depuración y pruebas:}}
\clearpage

\section{Resolución Problema 5}
\subsection{Problema:}
Dado un numero binario de $n$ bits regresar su equivalente en decimal.

\subsection{\textbf{Descripción del problema:}}

\subsection{\textbf{Definición de solución:}}

\subsection{\textbf{Diseño de la solución:}}

\subsection{\textbf{Desarrollo de la solución:}}

\subsection{\textbf{Depuración y pruebas:}}
\clearpage

\section{Resolución Problema 6}
\subsection{Problema:}
\item Dada una tabla de verdad de $n$ bits generar la expresión booleana que genere de manera fidedigna las salidas de esta tabla.

\subsection{\textbf{Descripción del problema:}}

\subsection{\textbf{Definición de solución:}}

\subsection{\textbf{Diseño de la solución:}}

\subsection{\textbf{Desarrollo de la solución:}}

\subsection{\textbf{Depuración y pruebas:}}
\clearpage


\section{CONCLUSION}
El manuscrito debe incluir direcciones futuras de la investigación. Se recomienda encarecidamente a los autores que no hagan referencia a varias figuras o tablas en la conclusión; estos deben mencionarse en el cuerpo del artículo.
\vspace*{-8pt}


\section{AGRADECIMIENTOS}
Queremos expresar nuestro sincero agradecimiento a aquellos que contribuyeron de manera significativa al desarrollo de nuestro proyecto integrador. En particular, deseamos reconocer y agradecer a los siguientes docentes por su invaluable apoyo y orientación:

Agradecemos al profesor Neri Pérez Giovany Humberto por su dedicación y asesoramiento en el aprendizaje de LaTeX, así como por su incansable labor en la corrección de la codificación y la explicación detallada de los problemas de cálculo que surgieron durante el desarrollo de nuestro proyecto.

Extendemos nuestro agradecimiento al profesor Agustín Soto Arista, cuya colaboración fue fundamental en las pruebas de los problemas relacionados con Matemáticas discretas. Su expertise y disposición para ayudarnos fueron esenciales para superar obstáculos y mejorar la calidad de nuestro trabajo.

Además, agradecemos a la profesora Eunice Santiago Manzano por su valiosa asistencia en los ensayos de exposición y su dedicación en la depuración de problemas. Su orientación y perspectiva fueron esenciales mejorar nuestra presentación y abordar eficientemente los desafíos que encontramos.

Estamos agradecidos por el apoyo constante de estos profesores, cuya contribución ha sido fundamental en el éxito de nuestro proyecto integrador.

\def\refname{REFERENCES}

\begin{thebibliography}{1}

\bibitem{AA1}
G. M. Amdahl, G. A. Blaauw, and F. P. Brooks, ``Architecture of the IBM System/360,'' {\it IBM J. Res. Dev}., vol. 8, no. 2, pp. 87--101, 1964. (Journal)

\bibitem{BB1}  
Cunoticias. (s. f.). Complemento a 1 (uno) con ejemplos. https://www.cunoticias.com/internet/complemento-a-1-uno-con-ejemplos.php


\bibitem{CC1}
Johnsonbaugh, R. (s. f.). MATEMÁTICAS DISCRETAS. Pearson Educación.

\bibitem{DD1}
Fernández, M. Y. A., Hernández, Y. J. S., & Montaña, M. M. (s. f.). Matemáticas Discretas:: Con un enfoque desde la ingeniería y ciencias sociales - Conceptos básicos. Editorial de la Universidad Pedagógica y Tecnológica de Colombia - UPTC.

\end{thebibliography}\vspace*{-8pt}


\begin{IEEEbiography}{Cuadros Romero Francisco Javier}{\,}Todas las biografías se limitan a un párrafo y deben ser muy sintéticas. Se puede agregar la carrera en la cual el estudiante esta enrolado. Se pueden mencionar los 3 intereses principales del estudiante. Asi como su aspiración en el corto y mediano plazo. Al final de la biografia de cada estudiante se debe agregar el enlace a su pagina personal en Github: 
%\vadjust{\vfill\pagebreak}
\end{IEEEbiography}


\begin{IEEEbiography}{Karen Perez Ortiz}{\,}Originaria de Mixquiahuala de Juarez, es una estudiante de Ingeniería en Tecnologías de la Información y Comunicaciones cuyo interés por la tecnología floreció durante la secundaria, gracias a una reveladora convención tecnológica. Fascinada por la versatilidad creativa que una computadora puede ofrecer, se sumerge en la exploración de diversas ramas tecnológicas. Más allá de su dedicación académica, disfruta de series animadas, la astronomía y seguir a streamers. Con metas ambiciosas, no solo aspira a concluir la universidad, sino también a dejar un legado significativo en el campo. Su enfoque se centra en el desarrollo web con énfasis en Experiencia de Usuario (UX), y su entusiasmo por la Inteligencia Artificial y el Aprendizaje Automático la impulsa a profundizar constantemente en estos campos, motivada por la perspectiva de contribuir al avance tecnológico. https://karenperezor.github.io/.
\end{IEEEbiography}

\end{document}

